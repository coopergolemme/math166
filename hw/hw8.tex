\documentclass[12pt]{article}

\usepackage{epsfig}
\usepackage{amsmath,amsthm,amssymb}
\usepackage{bbm}
\usepackage{graphicx}
\usepackage{float}
\usepackage{hyperref}




\usepackage{listings}
\usepackage{amsmath} % For the array environment


% These lines describe the environments we will use: Lemma and Theorem
% Although may sound repetitive to declare that our lemma will be displayed as Lemma
% it is needed so that LaTeX can write documents in other languages 
\newtheorem{lemma}{Lemma}
\newtheorem{theorem}{Theorem}

% The commands below change the bold text where it says "Section" into "Question"
\usepackage{titlesec}
\titleformat{\section}
{\normalfont\Large\bfseries}{Question~\thesection:}{1em}{}

%Commands below change page margins (this much space at the titlepage, etc)
\newlength{\toppush}
\setlength{\toppush}{2\headheight}
\addtolength{\toppush}{\headsep}

%Name and subject of the class
\def\subjnum{Math 166}
\def\subjname{Statistics}

%Name of the student, university name and which semester
\def\doheading#1#2#3{\vfill\eject\vspace*{-\toppush}%
  \vbox{\hbox to\textwidth{{\bf} \subjnum: \subjname \hfil Cooper Golemme}%
    \hbox to\textwidth{{\bf} Tufts University, Fall 2024 \hfil#3\strut}%
    \hrule}}

%Command for the title of the document (Homework 0)
\newcommand{\htitle}[1]{\vspace*{1.25ex plus 1ex minus 0ex}%
\begin{center}
{\large\bf #1}
\end{center}} 


%%%%%%%%%%%%%%%%%%%%%%%%%%%%%%%%%%%%%%%%%%%%%%%%%%%%%%%%%%%%%%%%%%%
% BEGIN DOCUMENT
%%%%%%%%%%%%%%%%%%%%%%%%%%%%%%%%%%%%%%%%%%%%%%%%%%%%%%%%%%%%%%%%%%%
\doheading{2}{title}{Homework 8}

\begin{document}

\section{Chapter 15 \#1}
Prove:
\begin{enumerate}
    \item \( Y \perp Z \).
    \item \( \psi = 1 \).
    \item \( \gamma = 0 \).
    \item For \( i, j \in \{0, 1\} \), \( p_{ij} = p_{i\cdot} p_{\cdot j} \).
\end{enumerate}



\pagebreak
\section{Chapter 15 \#4}
The \textit{New York Times} (January 8, 2003, page A12) reported the following data on death sentencing and race, from a study in Maryland:

\begin{center}
\begin{tabular}{|c|c|c|}
\hline
 & \textbf{Death Sentence} & \textbf{No Death Sentence} \\
\hline
\textbf{Black Victim} & 14 & 641 \\
\hline
\textbf{White Victim} & 62 & 594 \\
\hline
\end{tabular}
\end{center}

Analyze the data using the tools from this chapter. Interpret the results. Explain why, based only on this information, you can’t make causal conclusions. (The authors of the study did use much more information in their full report.)

\textbf{Answer:}

This problem can be represented as two binary random variables. We can say that the random variable $Y = 0$ if a death sentence is issued and $Y = 1$ if a death sentence is not issued and the random variable $Z = 0$ if the victim is black and $Z = 1$ if the victim is white.

Let $X = (X_{00}, X_{01}, X_{10}, X_{11})$ be the vector of counts as in the table below (where Y is the random variable for death sentence and Z for race)

\begin{center}
\begin{tabular}{|c|c|c|c|}
\hline
 & $Y = 0$ & $Y = 1$ & \\ 
\hline
$Z = 0$ & $X_{00}$ & $X_{01}$ & $X_{0\cdot}$ \\ 
\hline
$Z = 1$ & $X_{10}$ & $X_{11}$ & $X_{1\cdot}$ \\ 
\hline
 & $X_{\cdot 0}$ & $X_{\cdot 1}$ & $n = X_{\cdot\cdot}$ \\ 
\hline
\end{tabular}
\end{center}

We are interested in measuring the vector of counts $X$. We know that each entry in $X$ has a certain probability of occuring, which we can denote $p_{ij} = \mathbb{P}(Y = i, Z = j)$ and the vector $p = (p_{00}, p_{01}, p_{10}, p_{11})$

Thus, as $X$ is a vector of counts, distributed accordingly by the vector $p$, we can say $X \sim Multinomial(n, p)$. We can now use Pearson's $\chi^2$ test statistic for independence:
\[
    U = \sum_{i=0}^1\sum_{j=0}^1 \frac{(X_{ij} - E_{ij})^2}{E_{ij}}
\]

Where, $E_{ij} = \frac{X_{\cdot j}X_{i\cdot}}{n}$

\pagebreak

Filling in the data for the count table, we get:


\begin{center}
\begin{tabular}{|c|c|c|c|}
\hline
 & $Y = 0$ & $Y = 1$ & \\ 
\hline
$Z = 0$ & 14 & 641 & 655 \\ 
\hline
$Z = 1$ & 62 & 594 & 656 \\ 
\hline
 & 76 & 1235 & 1311 \\ 
\hline
\end{tabular}
\end{center}


% \begin{align}
%      & U = \sum_{i=0}^1\sum_{j=0}^1 \frac{(X_{ij} - 1)^2}{1}\\
%      & U = \sum_{i=0}^1\sum_{j=0}^1 (X_{ij} - 1)^2\\
%      & \sum_{i=0}^1\sum_{j=0}^1 (X_{ij} - 1)^2 = \sum_{i=0}^1 (X_{i0} - 1)^2 + (X_{i1} - 1)^2\\
%      & = (X_{00} - 1)^2 + (X_{10} - 1)^2 + (X_{01} - 1)^2 + (X_{11} - 1)^2 \\
%      & = (X_{00} - 1)^2 + (X_{10} - 1)^2 + (X_{01} - 1)^2 + (X_{11} - 1)^2 \\
% \end{align}

\end{document}
